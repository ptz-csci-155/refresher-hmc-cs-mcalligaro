\documentclass[a4paper]{exam}

\usepackage{amsmath}
\usepackage{esvect}
\usepackage{url} 


% Header and footer.
\pagestyle{headandfoot}
\runningheadrule
\runningfootrule
\runningheader{CSCI 155}{Refresher}{Spring 2019}
\runningfooter{}{Page \thepage\ of \numpages}{}
\firstpageheader{}{}{}

\qformat{{\large\bf Problem \thequestion. \thequestiontitle}\hfill}

\title{Refresher}
\author{CSCI 155 Computer Graphics\\Pitzer College}
\date{Spring 2019}

\begin{document}
\maketitle

This refresher provides an indication of what you are expected to know coming into this course. You may find in some of the questions below that required details are missing. If so, it is assumed that you are knowledgeable enough to make reasonable decisions based on the context and/or the provided information. You need not be on top of the math but it is expected that you have sufficient background to understand the pages that show up when you search online for help.

The refresher is not graded and does not have to be submitted. If there is something in it that you would like to discuss, I will be happy to do so in our meeting. You are welcome to post to the forum in the meanwhile.

\begin{questions}

  \titledquestion{Matrix Arithmetic}
  \begin{parts}
    \part Write down the 3x3 {\it identity matrix}.

	$$
    \left(\begin{array}{ccc}
  		1 & 0 & 0\\
    	0 & 1 & 0\\
   	 	0 & 0 & 1
    \end{array}\right)
	$$
    
    
    \part\label{transform} Given an $n\times n$ matrix, $A$, a vector, $x$, and the equation below
    \[ Ax = b, \]
    what are the dimensions of $x$ and $b$?  
      
    $x$ and $b$ are both $n$ dimensional vectors, meaning they have dimensions $n \times 1$.
    
    
    \part Write down an expression for $x$ in (\ref{transform}).
    
    Assuming that $A$ is invertible, $x = A^{-1} b$
    
    
    \part\label{eq} Write the following system of equations in matrix form.
    \begin{align*}
      2x + y - 2z	& =  3\\
      x - y - z & =	0\\
      x + y + 3z & = 12
    \end{align*}
    
	$$
     \left(\begin{array}{ccccc}
          2 & 1 & -2 & | & 3\\
          1 & -1 & -1 & | & 0\\
          1 & 1 & 3 & | & 12
    \end{array}\right)
	$$
    
    
    \part Solve the system of equations from (\ref{eq}).
    
    First, we perform elementary row operations (EROs) to transform the matrix into reduced row echelon form (RREF).
    
    $$
     \left(\begin{array}{ccccc}
          1 & 0 & 0 & | & \frac{7}{2}\\
          0 & 1 & 0 & | & 1\\
          0 & 0 & 1 & | & \frac{5}{2}
    \end{array}\right)
	$$
    
    We interpret this as $x = \frac{7}{2}, y = 1, z = \frac{5}{2}$. 
    
    
    
    \part Find the transpose and inverse of the matrix
    \[
      \left(
        \begin{array}{ccc}
          1 & 2 & 1\\
          1 & 1 & 1\\
          2 & 1 & 1
        \end{array}
      \right)
    \].
    
    We find the transpose by swaping the row and column coordinates for each entry.
	$$
     \left(\begin{array}{ccccc}
          1 & 1 & 2\\
          2 & 1 & 1\\
          1 & 1 & 1
    \end{array}\right)
	$$
	
	To find the inverse, we create an augmented matrix with the identity matrix and perform EROs to transform the first matrix into the identity matrix.  
	
	$$
     \left(\begin{array}{ccccccc}
          1 & 2 & 1 & | & 1 & 0 & 0\\
          1 & 1 & 1 & | & 0 & 1 & 0\\
          2 & 1 & 1 & | & 0 & 0 & 1
    \end{array}\right)
    \to
     \left(\begin{array}{ccccccc}
          1 & 0 & 0 & | & 0 & -1 & 0\\
          0 & 1 & 0 & | & 1 & -1 & 0\\
          0 & 0 & 1 & | & -1 & 3 & -1
    \end{array}\right)
	$$
	
	The inverse is the resulting right hand side matrix, namely
	$$
     \left(\begin{array}{ccc}
          0 & -1 & 0\\
          1 & -1 & 0\\
          -1 & 3 & -1
    \end{array}\right)
	$$
	
  \end{parts}

  \newpage
  \titledquestion{Vectors}
  Given the points $P(1,-2,0)$, $Q(3,1,4)$, and $R(0,-1,2)$, determine
  \begin{parts}
    \part the vectors $\vv{PQ}$ and $\vv{PR}$.
    $$\vv{PQ} = Q - P = (2,3,4)$$
    $$\vv{PR} = R - P = (-1,1,2)$$
    
    \part the angle between the vectors $\vv{PQ}$ and $\vv{PR}$.
    
    By the cosine formula, 
  	$$\theta = \cos^{-1}\left({\frac{\vv{PQ} \cdot \vv{PR}}{||\vv{PQ}|| * ||\vv{PR}||}}\right) = \cos^{-1}\left(\frac{9}{\sqrt{29} * \sqrt{6}}\right) \approx 0.8199 \text{ radians}$$
  	
  	  	
    \part a vector perpendicular to both $\vv{PQ}$ and $\vv{PR}$.
   	
   	The cross product of two vectors is perpendicular to both vectors.  Therefore, $\vv{PQ} \times \vv{PR} = (2,8,5)$ is perpendicular to both $\vv{PQ}$ and $\vv{PR}$.
    
    
    
    \part the length of the projection of $\vv{PR}$ on $\vv{PQ}$.
    
    We calculate the scalar projection of $\vv{PR}$ onto $\vv{PQ}$ as 
    $$\frac{\vv{PR} \cdot \vv{PQ}}{||\vv{PQ}||} = \frac{9}{\sqrt{29}} \approx 1.6713$$
    
  \end{parts}

  \newpage
  \titledquestion{Geometry}
  Given the points $P(1,-2,0)$, $Q(3,1,4)$, and $R(0,-1,2)$, determine the {\it vector}, {\it scalar}, and {\it parametric} equations of the following.
  \begin{parts}
    \part the ray starting at $P$ and passing through $Q$.
    \part the circle with center at $P$ and passing through $Q$.
    \part the plane containing the points $P$, $Q$, and $R$.
  \end{parts}

  \newpage
  \titledquestion{Vector Class}
  Implement a class, {\tt Vector}, in C++ to represent a 3D vector and define the following operations for it. Decide in each case whether the operation should be implemented as a method or an external function.
  \begin{parts}
    \part \underline{\tt dot}:  takes another {\tt Vector} object as a parameter and returns the dot product.
    \part \underline{\tt cross}: takes another {\tt Vector} object as a parameter and returns the cross product.
    \part \underline{\tt operator+}: takes another {\tt Vector} object as a parameter and returns the vector sum.
    \part \underline{\tt operator+=}: takes another {\tt Vector} object as a parameter and adds it to this {\tt Vector}.
    \part \underline{\tt operator-}: takes another {\tt Vector} object as a parameter and returns the vector difference.
    \part \underline{\tt operator-=}: takes another {\tt Vector} object as a parameter and subtracts it from the {\tt Vector}.
    \part \underline{\tt operator*}: takes a scalar as a parameter and returns the scaled vector. The operation must be commutative.
    \part \underline{\tt operator*=}: takes a scalar as a parameter and scales this {\tt Vector}.
    \part \underline{\tt norm}: returns the norm or magnitude of this {\tt Vector}.
    \part \underline{\tt normalize}: normalizes this {\tt Vector}.
    \part \underline{\tt operator-}: negates this {\tt Vector}.\\
    
	The corresponding code can be found in the {\tt vector} directory of this repository. 
  \end{parts}

\end{questions}

\section*{Credits}
Some of the questions have been adapted from:
\begin{itemize}
\item \url{https://math.dartmouth.edu/archive/m8s00/public_html/handouts/matrices3/node4.html}.
\item \url{http://tutorial.math.lamar.edu/Classes/CalcIII/EqnsOfPlanes.aspx}.
\end{itemize}

\end{document}
